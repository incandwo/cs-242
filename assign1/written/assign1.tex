
\documentclass[11pt]{article}

\input{../../tex/defs.tex}

% Useful syntax commands
\newcommand{\jarr}[1]{\left[#1\right]}   % \jarr{x: y} = {x: y}
\newcommand{\jobj}[1]{\left\{#1\right\}} % \jobj{1, 2} = [1, 2]
\newcommand{\pgt}[1]{\, > {#1}}          % \pgt{1} = > 1
\newcommand{\plt}[1]{\, < {#1}}          % \plt{2} = < 2
\newcommand{\peq}[1]{\, = {#1}}          % \peq{3} = = 3
\newcommand{\prop}[1]{\langle{#1}\rangle}% \prop{x} = <x>
\newcommand{\matches}[2]{{#1}\sim{#2}}   % \matches{a}{b} = a ~ b
\newcommand{\aeps}{\varepsilon}          % \apes = epsilon
\newcommand{\akey}[2]{.{#1}\,{#2}}       % \akey{s}{a} = .s a
\newcommand{\aidx}[2]{[#1]\,{#2}}        % \aidx{i}{a} = [i] a
\newcommand{\apipe}[1]{\mid {#1}}        % \apipe{a} = | a

% Other useful syntax commands:
%
% \msf{x} = x (not italicised)
% \falset = false
% \truet = true
% \tnum = num
% \tbool = bool
% \tstr = str


\begin{document}

\hwtitle
  {Assignment 1 resubmit}
  {Breno Dal Bianco (bdbianco)}

I am only typing Problem 2 Part 2 since I got full marks on the other parts, 
but I've attached them for reference.

\problem{Problem 2}

Part 2:

\begin{mathpar}
\ir{A-Val-Ep}{\ }{\matches{\aeps}{\tau}}

\ir{A-Val-Dict}
    {\matches{a}{\tau'}}
    {\matches{\akey{s}{a}}{\jobj{(s: \tau')*}}}

\ir{A-Val-Arr}
    {\matches{a}{\tau'}}
    {\matches{\jarr{n}a}{\jarr{\tau'}}}

\ir{A-Val-Pipe}
    {\matches{a}{\tau'}}
    {\matches{\apipe{a}}{\jarr{\tau'}}}

\end{mathpar}

\textit{Accessor safety}: for all $a, j, \tau$, if $\matches{a}{\tau}$ and $\matches{j}{\tau}$, then there exists a $j'$ such that $\evals{(a, j)}{(\aeps, j')}$.

\begin{proof}

Let $P(a) =$ for all $j, \tau$, if $\matches{a}{\tau}$ and $\matches{j}{\tau}$, then there exists a $j'$ such that $\evals{(a, j)}{(\aeps, j')}$.

First we will prove the base case $P(\aeps)$. Pick $j' = j$. Trivially, we have that $\evals{(\aeps, j)}{(\aeps, j')}$ in zero steps.

Now we will proceed by case, proving $P(a) \rightarrow P(\akey{s}{a})$, 
$P(a) \rightarrow P(\jarr{n}a)$, and $P(a) \rightarrow P(\apipe{a})$.

\begin{enumerate}
\item Assume $P(a)$. Pick an arbitrary acessor of the form $\akey{s}{a}$, an object
$j$ and a schema $\tau$ such that $\matches{\akey{s}{a}}{\tau}$ and $\matches{j}{\tau}$.
By inversion of A-Val-Dict, we have $\tau = \jobj{(s: \overline{\tau})*}$ and 
$\matches{a}{\overline{\tau}}$. Similarly, by inversion of S-Dict, since 
$\matches{j}{\jobj{(s: \overline{\tau})*}}$, then $j = \jobj{(s: \overline{j})*}$ and
$\matches{\overline{j}}{\overline{\tau}}$. 

So, we can see that 
$\steps{(\akey{s}{a}, j)}{(a, \overline{j})}$. Now, since $\matches{a}{\overline{\tau}}$ 
and $\matches{\overline{j}}{\overline{\tau}}$, then by $P(a)$ there is a $j'$ such that
$\evals{(a, \overline{j})}{(\aeps, j')}$. 

Thus, we have shown that
$\evals{(\akey{s}{a}, j)}{(\aeps, j')}$, so $P(a) \rightarrow P(\akey{s}{a})$.

\item Assume $P(a)$. Pick an arbitrary acessor of the form $\jarr{n}a$, an object
$j$ and a schema $\tau$ such that $\matches{\jarr{n}a}{\tau}$ and $\matches{j}{\tau}$.
By inversion of A-Val-Arr, we have $\tau = \jarr{\overline{\tau}}$ and 
$\matches{a}{\overline{\tau}}$. Similarly, by inversion of S-List, since 
$\matches{j}{\jarr{\overline{\tau}}}$, then $j = \jarr{~\overline{j}*}$ and
$\matches{\overline{j}}{\overline{\tau}}$. 

So, we can see that 
$\steps{(\jarr{n}a, j)}{(a, \overline{j})}$. Now, since $\matches{a}{\overline{\tau}}$ 
and $\matches{\overline{j}}{\overline{\tau}}$, then by $P(a)$ there is a $j'$ such that 
$\evals{(a, \overline{j})}{(\aeps, j')}$.

Thus, we have shown that
$\evals{(\jarr{n}a, j)}{(\aeps, j')}$, so $P(a) \rightarrow P(\jarr{n}a)$.

\item Assume $P(a)$. Pick an arbitrary acessor of the form $\apipe{a}$, an object
$j$ and a schema $\tau$ such that $\matches{\apipe{a}}{\tau}$ and $\matches{j}{\tau}$.
By inversion of A-Val-Pipe, we have $\tau = \jarr{\overline{\tau}}$ and 
$\matches{a}{\overline{\tau}}$. Similarly, by inversion of S-List, since 
$\matches{j}{\jarr{\overline{\tau}}}$, then $j = \jarr{~\overline{j}*}$ and
$\matches{\overline{j}}{\overline{\tau}}$. 

Since $\matches{a}{\overline{\tau}}$ and $\matches{\overline{j}}{\overline{\tau}}$, then 
by $P(a)$ there is a $j'$ such that $\evals{(a, \overline{j})}{(\aeps, j')}$ in $k$ steps.

Therefore, we can apply A-pipe $k$ times and we get 
$\evals{(\apipe{a}, \jarr{~\overline{j}*})}{(\apipe{a_k}, \overline{j_k})}$ with 
$a_k = \aeps$. At this point, all that is left is to apply A-pipe-epsilon to get rid of the pipe operator and we get 
$\steps{(\apipe{a_k}, \overline{j_k})}{(\aeps, \overline{j_k})}$.

Thus, we have shown that
$\evals{(\apipe{a}, j)}{(\aeps, j')}$, so $P(a) \rightarrow P(\apipe{a})$.
\end{enumerate}{}

We have proven $P(a) \rightarrow P(\akey{s}{a})$, 
$P(a) \rightarrow P(\jarr{n}a)$, and $P(a) \rightarrow P(\apipe{a})$, which cover all
of our accessor grammar, so we've shown $P(a)$ holds for all any accessor $a$, as
intended.

\end{proof}

\end{document}
