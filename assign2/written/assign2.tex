\documentclass[11pt]{article}

%%% PACKAGES

\usepackage{fullpage}  % 1 page margins
\usepackage{latexsym}  % extra symbols, e.g. \leadsto
\usepackage{verbatim}  % verbatim mode
\usepackage{proof}     % proof mode
\usepackage{amsthm,amssymb,amsmath}  % various math symbols
\usepackage{color}     % color control
\usepackage{etoolbox}  % misc utilities
\usepackage{graphics}  % images
\usepackage{mathpartir}% inference rules
\usepackage{hyperref}  % hyperlinks
\usepackage{titlesec}  % title/section controls
\usepackage{minted}    % code blocks
\usepackage{tocstyle}  % table of contents styling
\usepackage[hang,flushmargin]{footmisc} % No indent footnotes
\usepackage{parskip}   % no parindent
\usepackage{tikz}      % drawing figures
\usepackage[T1]{fontenc}
%%% COMMANDS

% Typography and symbols
\newcommand{\msf}[1]{\mathsf{#1}}
\newcommand{\ctx}{\Gamma}
\newcommand{\qamp}{&\quad}
\newcommand{\qqamp}{&&\quad}
\newcommand{\Coloneqq}{::=}
\newcommand{\proves}{\vdash}
\newcommand{\str}[1]{{#1}^{*}}
\newcommand{\eps}{\varepsilon}
\newcommand{\brc}[1]{\{{#1}\}}
\newcommand{\binopm}[2]{{#1}~\bar{\oplus}~{#2}}
\newcommand{\aequiv}{\equiv_\alpha}

% Untyped lambda calculus
\newcommand{\fun}[2]{\lambda ~ {#1} ~ . ~ {#2}}
\newcommand{\app}[2]{{#1} ~ {#2}}
\newcommand{\fix}[1]{\msf{fix}~{#1}}

% Typed lambda calculus - expressions
\newcommand{\funt}[3]{\lambda ~ ({#1} : {#2}) ~ . ~ {#3}}
\newcommand{\ift}[3]{\msf{if} ~ {#1} ~ \msf{then} ~ {#2} ~ \msf{else} ~ {#3}}
\newcommand{\rec}[5]{\msf{rec}(#1; ~ #2.#3.#4)(#5)}
\newcommand{\lett}[4]{\msf{let} ~ \hasType{#1}{#2} = {#3} ~ \msf{in} ~ {#4}}
\newcommand{\falset}{\msf{false}}
\newcommand{\truet}{\msf{true}}
\newcommand{\case}[5]{\msf{case} ~ {#1} ~ \{ L({#2}) \to {#3} \mid R({#4}) \to {#5} \}}
\newcommand{\pair}[2]{({#1},{#2})}
\newcommand{\proj}[2]{{#1} . {#2}}
\newcommand{\inj}[3]{\msf{inj} ~ {#1} = {#2} ~ \msf{as} ~ {#3}}
\newcommand{\letv}[3]{\msf{let} ~ {#1} = {#2} ~ \msf{in} ~ {#3}}
\newcommand{\fold}[2]{\msf{fold}~{#1}~\msf{as}~{#2}}
\newcommand{\unfold}[1]{\msf{unfold}~{#1}}
\newcommand{\poly}[2]{\Lambda~{#1}~.~{#2}}
\newcommand{\polyapp}[2]{{#1}~[{#2}]}

% Typed lambda calculus - types
\newcommand{\tnum}{\msf{number}}
\newcommand{\tstr}{\msf{string}}
\newcommand{\tint}{\msf{int}}
\newcommand{\tbool}{\msf{bool}}
\newcommand{\tfun}[2]{{#1} \rightarrow {#2}}
\newcommand{\tprod}[2]{{#1} \times {#2}}
\newcommand{\tsum}[2]{{#1} + {#2}}
\newcommand{\trec}[2]{\mu~{#1}~.~{#2}}
\newcommand{\tvoid}{\msf{void}}
\newcommand{\tunit}{\msf{unit}}
\newcommand{\tpoly}[2]{\forall~{#1}~.~{#2}}

% WebAssembly
\newcommand{\wconst}[1]{\msf{i32.const}~{#1}}
\newcommand{\wbinop}[1]{\msf{i32}.{#1}}
\newcommand{\wgetlocal}[1]{\msf{get\_local}~{#1}}
\newcommand{\wsetlocal}[1]{\msf{set\_local}~{#1}}
\newcommand{\wgetglobal}[1]{\msf{get\_global}~{#1}}
\newcommand{\wsetglobal}[1]{\msf{set\_global}~{#1}}
\newcommand{\wload}{\msf{i32.load}}
\newcommand{\wstore}{\msf{i32.store}}
\newcommand{\wsize}{\msf{memory.size}}
\newcommand{\wgrow}{\msf{memory.grow}}
\newcommand{\wunreachable}{\msf{unreachable}}
\newcommand{\wblock}[1]{\msf{block}~{#1}}
\newcommand{\wloop}[1]{\msf{loop}~{#1}}
\newcommand{\wbr}[1]{\msf{br}~{#1}}
\newcommand{\wbrif}[1]{\msf{br\_if}~{#1}}
\newcommand{\wreturn}{\msf{return}}
\newcommand{\wcall}[1]{\msf{call}~{#1}}
\newcommand{\wlabel}[2]{\msf{label}~\{#1\}~{#2}}
\newcommand{\wframe}[2]{\msf{frame}~({#1}, {#2})}
\newcommand{\wtrapping}{\msf{trapping}}
\newcommand{\wbreaking}[1]{\msf{breaking}~{#1}}
\newcommand{\wreturning}[1]{\msf{returning}~{#1}}
\newcommand{\wconfig}[5]{\{\msf{module}{:}~{#1};~\msf{mem}{:}~{#2};~\msf{locals}{:}~{#3};~\msf{stack}{:}~{#4};~\msf{instrs}{:}~{#5}\}}
\newcommand{\wfunc}[4]{\{\msf{params}{:}~{#1};~\msf{locals}{:}~{#2};~\msf{return}~{#3};~\msf{body}{:}~{#4}\}}
\newcommand{\wmodule}[1]{\{\msf{funcs}{:}~{#1}\}}
\newcommand{\wcg}{\msf{globals}}
\newcommand{\wcf}{\msf{funcs}}
\newcommand{\wci}{\msf{instrs}}
\newcommand{\wcs}{\msf{stack}}
\newcommand{\wcl}{\msf{locals}}
\newcommand{\wcm}{\msf{mem}}
\newcommand{\wcmod}{\msf{module}}
\newcommand{\wsteps}[2]{\steps{\brc{#1}}{\brc{#2}}}
\newcommand{\with}{\underline{\msf{with}}}
\newcommand{\wvalid}[2]{{#1} \vdash {#2}~\msf{valid}}
\newcommand{\wif}[2]{\msf{if}~{#1}~{\msf{else}}~{#2}}
\newcommand{\wfor}[4]{\msf{for}~(\msf{init}~{#1})~(\msf{cond}~{#2})~(\msf{post}~{#3})~{#4}}
% assign4.3 custom
\newcommand{\wtry}[2]{\msf{try}~{#1}~\msf{catch}~{#2}}
\newcommand{\wraise}{\msf{raise}}
\newcommand{\wraising}[1]{\msf{raising}~{#1}}

% Inference rules
%\newcommand{\inferrule}[3][]{\cfrac{#2}{#3}\;{#1}}
\newcommand{\ir}[3]{\inferrule*[right=\text{(#1)}]{#2}{#3}}
\newcommand{\s}{\hspace{1em}}
\newcommand{\nl}{\\[2em]}
\newcommand{\steps}[2]{#1 \boldsymbol{\mapsto} #2}
\newcommand{\evals}[2]{#1 \boldsymbol{\overset{*}{\mapsto}} #2}
\newcommand{\subst}[3]{[#1 \rightarrow #2] ~ #3}
\newcommand{\dynJ}[2]{#1 \proves #2}
\newcommand{\dynJC}[1]{\dynJ{\ctx}{#1}}
\newcommand{\typeJ}[3]{#1 \proves \hasType{#2}{#3}}
\newcommand{\typeJC}[2]{\typeJ{\ctx}{#1}{#2}}
\newcommand{\hasType}[2]{#1 : #2}
\newcommand{\val}[1]{#1~\msf{val}}
\newcommand{\num}[1]{\msf{Int}(#1)}
\newcommand{\err}[1]{#1~\msf{err}}
\newcommand{\trans}[2]{#1 \leadsto #2}
\newcommand{\size}[1]{|#1|}

\newcommand{\hwtitle}[2]{\begin{center}{\Large #1} \\[0.5em] {\large #2}\end{center}\vspace{1em}}
\newcommand{\toc}{{\hypersetup{hidelinks}\tableofcontents}}
\newcommand{\problem}[1]{\section*{#1}}

%%% CONFIG

% Spacing around title/sections
\titlelabel{\thetitle.\quad}
\titlespacing*{\section}{0pt}{10pt}{0pt}
\titlespacing*{\subsection}{0pt}{10pt}{0pt}


% Show color on hyperlinks
\hypersetup{colorlinks=true}

% Stylize code blocks
\usemintedstyle{xcode}
% TODO(wcrichto): remove line of spacing beneath code blocks
\newcommand{\nm}[2]{
  \newminted{#1}{#2}
  \newmint{#1}{#2}
  \newmintinline{#1}{#2}}
\nm{lua}{}
\nm{ocaml}{}
\nm{rust}{}
\nm{prolog}{}

\newcommand{\ml}[1]{\ocamlinline|#1|}

\usetocstyle{standard}


\begin{document}

\hwtitle
  {Assignment 2}
  {Breno Dal Bianco (bdbianco)} %% REPLACE THIS WITH YOUR NAME/ID

\problem{Problem 2}
\textbf{Part 1:}

\begin{mathpar}
\text{Step 1:}\qquad
\ir{D-App-Body}
  {\ir{D-App-Lam}
    {\ir{D-App-Done}
      {\ir{D-Lam}{ \ }{\val{\fun{\_}{x}}}}
      {\dynJ{\{x \rightarrow D\}}{\steps
        {\app{(\fun{x}{\fun{\_}{x}})}{L}}
        {\fun{\_}{x}}}}}
    {\dynJ{\{x \rightarrow D\}}{\steps
      {\app{\app{(\fun{x}{\fun{\_}{x}})}{L}}{*}}
      {\app{(\fun{\_}{x})}{*}}}}}
  {\dynJ{\varnothing}{\steps
    {\app{(\fun{x}{\app{\app{(\fun{x}{\fun{\_}{x}})}{L}}{*}})}{D}}
    {\app{(\fun{x}{\app{(\fun{\_}{x})}{*}})}{D}}}}

\text{Step 2:}\qquad
\ir{D-App-Body}
  {\ir{D-App-Body}
    {\ir{D-Var}
      {\ir{Set Theory}{ \ }{\steps x D \in \{x \rightarrow D, \_ \rightarrow *\}{}}
      }
      {\dynJ{\{x \rightarrow D, \_ \rightarrow *\}}{\steps
        x
        D
        }}}
    {\dynJ{\{x \rightarrow D\}}{\steps
      {\app{(\fun{\_}{x})}{*}}
      {\app{(\fun{\_}{D})}{*}}
      }}}
  {\dynJ{\varnothing}{\steps
    {\app{(\fun{x}{\app{(\fun{\_}{x})}{*}})}{D}}
    {\app{(\fun{x}{\app{(\fun{\_}{D})}{*}})}{D}}
    }}

\text{Step 3:}\qquad
\ir{D-App-Body}
  {\ir{D-App-Done}
    {\ir{D-Val}
      {\ }
      {\val{D}}}
    {\dynJ{\{x \rightarrow D\}}{\steps
      {\app{(\fun{\_}{D})}{*}}
      D
      }}}
  {\dynJ{\varnothing}{\steps
    {\app{(\fun{x}{\app{(\fun{\_}{D})}{*}})}{D}}
    {\app{(\fun{x}{D})}{D}}
    }}

\text{Step 4:}\qquad
  \ir{D-App-Done}
    {\ir{D-Val}
      {\ }
      {\val{D}}}
    {\dynJ{\varnothing}{\steps
      {\app{(\fun{x}{D})}{D}}
      D
    }}

\end{mathpar}

\textbf{Part 2:}

To do this, we'll rewrite D-App-Body and D-App-Done to provide the same functionality for the let construct.

\begin{mathpar}

\ir{D-App-Body}
  {\dynJ{\ctx, x \rightarrow e_\msf{var}} {\steps{e_\msf{body}}{e_\msf{body}'}}}
  {\dynJC{
    \steps
    {\msf{let} ~ x = {e_\msf{var}} ~ \msf{in} ~ {e_\msf{body}}}
    {\msf{let} ~ x = {e_\msf{var}} ~ \msf{in} ~ {e_\msf{body}'}}
  }} \s

\ir{D-App-Done}
  {\val{e_\msf{body}}}
  {\dynJC{
    \steps
      {\msf{let} ~ x = {e_\msf{var}} ~ \msf{in} ~ {e_\msf{body}}}
      {e_\msf{body}}}}
      
\end{mathpar}

\newpage

\problem{Problem 3}
\textbf{Part 1:}

D-Let violates preservation.

$Proof.$
Let $e = (\lett{x}{\msf{num}}{'hello'}{x})$. We can see that by D-Let, we can
step this to $e' =\ 'hello'$. Preservation states that if $\steps{e}{e'}$ then $e$ and $e'$ must have the same type $\tau$. We can see that 
\begin{mathpar}
\ir{T-Var} { \ }
{
  \ir{T-Let}
    {\typeJ{ \{\hasType{x}{\msf{num}\} }}{x}{\msf{num}}}
    {\typeJ{\varnothing}{(\lett{x}{\msf{num}}{'hello'}{x})}{\msf{num}}}
}
\end{mathpar}
so we have that $\hasType{e}{num}$. However, it is easy to see that $\hasType{e'}{\msf{string}}$. So we have that $\steps{e}{e'}$. but $e$ and $e'$ have different types, which violates preservation.

The problem is D-Let does not enforce that $\hasType{e_\msf{var}}{\tau_{\msf{var}}}$.
In the substitution above, if it were true that $\hasType{x}{\msf{num}}$, then preservation would be mantained. D-Let needs to require that we only substitute something for a variable which has the same type as the variable.

\textbf{Part 2.1:}

We will prove preservation holds for $\msf{rec}$.

$Proof.$ By rule induction on the dynamic semantics.

\begin{enumerate}
    \item D-Rec-Step: Assume 
    $\hasType{\rec{e_\msf{base}}{x_\msf{num}}{x_\msf{acc}}{e_\msf{acc}}{e_\msf{arg}}}{\tau}$
    and 
    $\steps{\rec{e_\msf{base}}{x_\msf{num}}{x_\msf{acc}}{e_\msf{acc}}{e_\msf{arg}}}
    {\rec{e_\msf{base}}{x_\msf{num}}{x_\msf{acc}}{e_\msf{acc}}{e_\msf{arg}'}}$
    . 
    This requires that $\steps{e_\msf{arg}}{e_\msf{arg}'}$. Then, from the IH and 
    $\hasType{e_\msf{arg}}{\msf{num}}$, we know that $\hasType{e_\msf{arg}'}{\msf{num}}$. 
    
    Inverting T-Rec for 
    $\hasType{\rec{e_\msf{base}}{x_\msf{num}}{x_\msf{acc}}{e_\msf{acc}}{e_\msf{arg}}}{\tau}$
    yields $\hasType{e_\msf{arg}}{\msf{num}}$, $\hasType{e_\msf{base}}{\tau}$, and 
    $\hasType{e_\msf{acc}}{\tau}$. Combining those with
    $\hasType{e_\msf{arg}'}{\msf{num}}$ and reapplying T-Rec gives 
    $\hasType{\rec{e_\msf{base}}{x_\msf{num}}{x_\msf{acc}}{e_\msf{acc}}{e_\msf{arg}'}}{\tau}$
    which satisfies preservation.
    
    \item D-Rec-Base: Assume
    $\hasType{\rec{e_\msf{base}}{x_\msf{num}}{x_\msf{acc}}{e_\msf{acc}}{0}}{\tau}$ and
    $\steps{\rec{e_\msf{base}}{x_\msf{num}}{x_\msf{acc}}{e_\msf{acc}}{0}}{e_\msf{base}}$.
    
    Inverting T-Rec for
    $\hasType{\rec{e_\msf{base}}{x_\msf{num}}{x_\msf{acc}}{e_\msf{acc}}{0}}{\tau}$
    yields $\hasType{e_\msf{base}}{\tau}$, and thus preservation holds.
    
    \item D-Rec-Dec: Assume
    $\hasType{\rec{e_\msf{base}}{x_\msf{num}}{x_\msf{acc}}{e_\msf{acc}}{n}}{\tau}$ and
    $\steps
      {\rec{e_\msf{base}}{x_\msf{num}}{x_\msf{acc}}{e_\msf{acc}}{n}}
      {[x_\msf{num} \rightarrow n, x_\msf{acc} \rightarrow \rec{e_\msf{base}}{x_\msf{num}}{x_\msf{acc}}{e_\msf{acc}}{n-1}] \ e_\msf{acc}}$.
    
    Inverting T-Rec for
    $\hasType{\rec{e_\msf{base}}{x_\msf{num}}{x_\msf{acc}}{e_\msf{acc}}{n}}{\tau}$
    yields $\hasType{e_\msf{base}}{\tau}$ and $\hasType{e_\msf{acc}}{\tau}$.
    
    Then, since $\hasType{(n-1)}{\msf{num}}$ we can reapply T-Rec and get that
    $\hasType{\rec{e_\msf{base}}{x_\msf{num}}{x_\msf{acc}}{e_\msf{acc}}{n-1}}{\tau}$.
    
    So we know $\hasType{n}{\msf{num}}$ and 
    $\hasType{\rec{e_\msf{base}}{x_\msf{num}}{x_\msf{acc}}{e_\msf{acc}}{n-1}}{\tau}$.
    Then, by the substitution typing lemma, we can conclude that
    $\hasType{{[x_\msf{num} \rightarrow n, x_\msf{acc} \rightarrow 
    \rec{e_\msf{base}}{x_\msf{num}}{x_\msf{acc}}{e_\msf{acc}}{n-1}] \ e_\msf{acc}}}{\tau}$
    , and thus preservation holds.

\end{enumerate}{}

Preservation therefore holds for all rules in the $\msf{rec}$ extension, 
which is we wanted to prove. \qed

\textbf{Part 2.2:}

We will prove progress holds for $\msf{rec}$.

$Proof.$ By rule induction on the static semantics.

Assume $e = \rec{e_\msf{base}}{x_\msf{num}}{x_\msf{acc}}{e_\msf{acc}}{e_\msf{arg}}$
and $\hasType{e}{\tau}$. Inverting T-Rec for $e$ yields 
$\hasType{e_\msf{arg}}{\msf{num}}$, $\hasType{e_\msf{base}}{\tau}$, and 
$\hasType{e_\msf{acc}}{\tau}$.

We will now case on the possible rules that yield $\hasType{e_\msf{arg}}{\msf{num}}$.

\begin{enumerate}
    \item T-Num: For T-Num to apply, we must have that $e_\msf{arg} = n$. Then, either
    $n = 0$ or $n \neq 0$.
    
    $n = 0$: In this case, we can apply D-Rec-Base and we have that
    $\steps{e}{e_\msf{base}}$, so there is progress.
    
    $n \neq 0$: In this case, we can apply D-Rec-Dec 
    and we have that $\steps{e}{[x_\msf{num} \rightarrow n, x_\msf{acc} \rightarrow \rec{e_\msf{base}}{x_\msf{num}}{x_\msf{acc}}{e_\msf{acc}}{n-1}] \ e_\msf{acc}}$,
    so there is also progress.
    
    \item T-Binop: In this case $e_\msf{arg} = e_L \oplus e_R$ with 
    $\hasType{e_L}{\msf{num}}$ and $\hasType{e_R}{\msf{num}}$, by inversion of T-Binop.
    By the IH, can assume progress for $e_L$ and $e_R$. This gives us a few possibilities:
    
    $\steps{e_L}{e_L'}$: Then, by D-Binop-L, $\steps{e_\msf{arg}}{e_L' \oplus e_R}$, and we can can apply D-Rec-Step, so ${\steps
    {\rec{e_\msf{base}}{x_\msf{num}}{x_\msf{acc}}{e_\msf{acc}}{e_\msf{arg}}}
    {\rec{e_\msf{base}}{x_\msf{num}}{x_\msf{acc}}{e_\msf{acc}}{e_L' \oplus e_R}}}$ and there is progress.
    
    $\val{e_L}$ and $\steps{e_R}{e_R'}$: Then, by D-Binop-R, $\steps{e_\msf{arg}}{e_L \oplus e_R'}$, and we can can apply D-Rec-Step, so ${\steps
    {\rec{e_\msf{base}}{x_\msf{num}}{x_\msf{acc}}{e_\msf{acc}}{e_\msf{arg}}}
    {\rec{e_\msf{base}}{x_\msf{num}}{x_\msf{acc}}{e_\msf{acc}}{e_L \oplus e_R'}}}$ and there is progress.
    
    $\val{e_L}$ and $\val{e_R}$: Then, by inversion of D-Num we know $e_L = n_L$ and
    $e_R = n_R$. Therefore, by D-Binop-Op $\steps{e_\msf{arg}}{n'}$ where 
    $n' = n_L \oplus n_R$, and we can can apply D-Rec-Step, so ${\steps
    {\rec{e_\msf{base}}{x_\msf{num}}{x_\msf{acc}}{e_\msf{acc}}{e_\msf{arg}}}
    {\rec{e_\msf{base}}{x_\msf{num}}{x_\msf{acc}}{e_\msf{acc}}{n'}}}$ and there is progress.

\end{enumerate}{}

We've shown that there is progress in every case, so progress holds for $\msf{rec}$, 
as desired. \qed

\end{document}
